
@techreport{bundeskriminalamt2020,
  title = {{Polizeiliche Kriminalstatistik 2019}},
  author = {Bundeskriminalamt},
  year = {2020},
  pages = {76},
  institution = {{Bundesministerium f{\"u}r Inneres}},
  file = {/Users/benjaminjoham/Zotero/storage/7W7MJML4/Broschuere_PKS_2019.pdf},
}

@misc{kachel2008,
  title = {{CSS / XSS \textendash{} Angriff (Cross Site Scripting) - eine Analyse}},
  author = {Kachel, Erich},
  year = {2008},
  month = aug,
  abstract = {Den vollst{\"a}ndigen 1. Teil des Dokumentes unter Creative Common-Lizenz als PDF-Datei herunterladen: Analyse und Ma{\ss}nahmen gegen Sicherheitsschwachstellen bei der Implementierung von Webanwendungen in PHP/MySQL (Teil 1) Cross Site Scripting beschreibt das Ausf{\"u}hren von Skriptbefehlen {\"u}ber unterschiedliche Webseiten hinweg. Praktisch bedeutet...},
  file = {/Users/benjaminjoham/Zotero/storage/R472DNC4/css-xss-–-angriff-cross-site-scripting.html},
  journal = {Most Things Web},
  note = {Library Catalog: www.erich-kachel.de}
}

@techreport{kpmg2020,
  title = {{Cyber Security in {\"O}sterreich}},
  author = {KPMG},
  year = {2020},
  pages = {67},
  institution = {{KPMG Security Services GmbH}},
  file = {/Users/benjaminjoham/Zotero/storage/WG663CLX/Cyber Security in Österreich.pdf},
}

@article{liu2019,
  title = {A {{Survey}} of {{Exploitation}} and {{Detection Methods}} of {{XSS Vulnerabilities}}},
  author = {Liu, Miao and Zhang, Boyu and Chen, Wenbin and Zhang, Xunlai},
  year = {2019},
  volume = {7},
  pages = {182004--182016},
  issn = {2169-3536},
  doi = {10.1109/ACCESS.2019.2960449},
  abstract = {As web applications become more prevalent, web security becomes more and more important. Cross-site scripting vulnerability abbreviated as XSS is a kind of common injection web vulnerability. The exploitation of XSS vulnerabilities can hijack users' sessions, modify, read and delete business data of web applications, place malicious codes in web applications, and control victims to attack other targeted servers. This paper discusses classification of XSS, and designs a demo website to demonstrate attack processes of common XSS exploitation scenarios. The paper also compares and analyzes recent research results on XSS detection, divides them into three categories according to different mechanisms. The three categories are static analysis methods, dynamic analysis methods and hybrid analysis methods. The paper classifies 30 detection methods into above three categories, makes overall comparative analysis among them, lists their strengths and weaknesses and detected XSS vulnerability types. In the end, the paper explores some ways to prevent XSS vulnerabilities from being exploited.},
  file = {/Users/benjaminjoham/Zotero/storage/7PA3KF9J/Liu et al. - 2019 - A Survey of Exploitation and Detection Methods of .pdf;/Users/benjaminjoham/Zotero/storage/79CDM2WE/8935148.html},
  journal = {IEEE Access},
  keywords = {cross-site scripting vulnerability,dynamic analysis methods,hybrid analysis methods,injection web vulnerability,Internet,program diagnostics,security of data,static analysis methods,Vulnerability detection,vulnerability exploitation,web applications,web security,XSS,XSS detection,XSS exploitation scenarios,XSS vulnerability types},
  note = {Conference Name: IEEE Access}
}

@misc{makarem2018,
  title = {{{DOM}}-{{Based Cross Site Scripting}} ({{DOM}}-{{XSS}})},
  author = {Makarem, Christopher},
  year = {2018},
  month = nov,
  abstract = {DOM-based XSS is a variant of both persistent and reflected XSS. In a DOM-based XSS attack, the malicious string is not actually parsed by\ldots{}},
  file = {/Users/benjaminjoham/Zotero/storage/3XYN2S6I/dom-based-cross-site-scripting-dom-xss-3396453364fd.html},
  howpublished = {https://medium.com/iocscan/dom-based-cross-site-scripting-dom-xss-3396453364fd},
  journal = {Medium},
  note = {Library Catalog: medium.com}
}

@misc{makarem2018a,
  title = {Reflected {{Cross Site Scripting}} (r-{{XSS}})},
  author = {Makarem, Christopher},
  year = {2018},
  month = nov,
  abstract = {Reflected XSS attacks, also known as non-persistent attacks, occur when a malicious script is reflected off of a web application to the\ldots{}},
  file = {/Users/benjaminjoham/Zotero/storage/AJ7N4E2B/reflected-cross-site-scripting-r-xss-b06c3e8d638a.html},
  howpublished = {https://medium.com/iocscan/reflected-cross-site-scripting-r-xss-b06c3e8d638a},
  journal = {Medium},
  note = {Library Catalog: medium.com}
}

@misc{makarem2018b,
  title = {Persistent {{Cross Site Scripting}} (p-{{XSS}})},
  author = {Makarem, Christopher},
  year = {2018},
  month = nov,
  abstract = {Cross Site Scripting (XSS) is a dangerously common code injection attack that allows an attacker to execute malicious JavaScript code in a\ldots{}},
  file = {/Users/benjaminjoham/Zotero/storage/UXSTYJ9G/persistent-cross-site-scripting-p-xss-557c70377554.html},
  howpublished = {https://medium.com/iocscan/persistent-cross-site-scripting-p-xss-557c70377554},
  journal = {Medium},
  note = {Library Catalog: medium.com}
}

@misc{ptsecurity2019,
  title = {Web {{Applications}} Vulnerabilities and Threats: Statistics for 2019},
  shorttitle = {Web {{Applications}} Vulnerabilities and Threats},
  author = {{ptsecurity}},
  year = {2019},
  abstract = {The percentage of web applications containing high-risk vulnerabilities in 2019 fell significantly, by 17 percentage points compared to the prior year. The average number of severe vulnerabilities per web application also fell, by almost one third. The last five years show a reduction in the percentage of sites containing severe vulnerabilities. This is an encouraging sign consistent with an overall improvement in security. The most commonly encountered web application vulnerabilities in 2019 involved Security Misconfiguration. One out of every five tested applications contained vulnerabilities allowing the hackers to attack a user session.},
  file = {/Users/benjaminjoham/Zotero/storage/F6ZSYWTI/web-vulnerabilities-2020.html},
  howpublished = {https://www.ptsecurity.com/ww-en/analytics/web-vulnerabilities-2020/},
  note = {Library Catalog: www.ptsecurity.com}
}

@article{chaudhari2014,
  title = {A {{Survey}} on {{Security}} and {{Vulnerabilities}} of {{Web Application}}},
  author = {Chaudhari, Gopal R. and Vaidya, Madhav V.},
  year = {2014},
  volume = {5},
  pages = {1856--1860},
  issn = {0975-1860},
  abstract = {Web applications are the distributed platform used for information sharing and services over Internet today. They are increasingly used for the financial, government, healthcare and many critical services. Modern web applications frequently implements the complex structure requires for user to perform actions in given order. The popularity adds value to these applications, which attracts attackers towards them. The attackers are well known about the valuable information accessible through the web application, which leads to serious security attacks on web applications. In this paper we survey the state of the art in web application security; first we explain working of web application and focus on the challenges for building secure web application. We organized the existing security vulnerabilities into the security properties that web application should preserved, discussed the root cause of these vulnerabilities and their corresponding preventive measures. Next we focus on the malware attacks on the web application, how the web applications are compromises for security and get infected by malware. Finally we summarize the lessons and discussed the future scope and opportunities in this area. Keywords\textemdash{} Security vulnerability, Web Application, separated by comma.},
  file = {/Users/benjaminjoham/Zotero/storage/MQJWRKJX/Chaudhari and Vaidya - 2014 - A Survey on Security and Vulnerabilities of Web Ap.pdf},
  journal = {IJCSIT},
  keywords = {Biologic Preservation,Internet,Malware,Penetration test,Threat (computer),Vulnerability (computing),Web application security},
  number = {2}
}

@inproceedings{dolnak2017,
  title = {Content {{Security Policy}} ({{CSP}}) as Countermeasure to {{Cross Site Scripting}} ({{XSS}}) Attacks},
  booktitle = {2017 15th {{International Conference}} on {{Emerging eLearning Technologies}} and {{Applications}} ({{ICETA}})},
  author = {Doln{\'a}k, Ivan},
  year = {2017},
  month = oct,
  pages = {1--4},
  doi = {10.1109/ICETA.2017.8102476},
  abstract = {This article presents one of several HTTP Security Headers - Content Security Policy (CSP) header, nowadays preferred countermeasure to Cross Site Scripting (XSS) attacks. It is emphasized that implementation of CSP header is relatively simple method how to improve security level of communication among people and devices over Internet. In the environment where secure web services are vital - web applications and Internet of Things (loT) networks, this way it is possible to strictly define communication parties and assets used in web services. Simple and effective reporting is native part of the CSP design, what means administrators can be notified about running attacks almost instantly. It is demonstrated on practical examples what benefits of CSP implementation can bring to communication and how easy is to propagate CSP header to web browser.},
  file = {/Users/benjaminjoham/Zotero/storage/8TFGE9S5/Dolnák - 2017 - Content Security Policy (CSP) as countermeasure to.pdf;/Users/benjaminjoham/Zotero/storage/JZVPGLLA/8102476.html},
  keywords = {Browsers,computer network security,Content Security Policy header,Cross Site Scripting attacks,CSP design,CSP header,CSP implementation,HTTP Security Headers,hypermedia,Internet of Things,Protocols,secure web services,security level,transport protocols,web applications,Web servers,Web services,Web sites,XSS}
}

@inproceedings{fang2018,
  title = {{{DeepXSS}}: {{Cross Site Scripting Detection Based}} on {{Deep Learning}}},
  shorttitle = {{{DeepXSS}}},
  booktitle = {Proceedings of the 2018 {{International Conference}} on {{Computing}} and {{Artificial Intelligence}} - {{ICCAI}} 2018},
  author = {Fang, Yong and Li, Yang and Liu, Liang and Huang, Cheng},
  year = {2018},
  pages = {47--51},
  publisher = {{ACM Press}},
  address = {{Chengdu, China}},
  doi = {10.1145/3194452.3194469},
  abstract = {Nowadays, Cross Site Scripting (XSS) is one of the major threats to Web applications. Since it's known to the public, XSS vulnerability has been in the TOP 10 Web application vulnerabilities based on surveys published by the Open Web Applications Security Project (OWASP). How to effectively detect and defend XSS attacks are still one of the most important security issues. In this paper, we present a novel approach to detect XSS attacks based on deep learning (called DeepXSS). First of all, we used word2vec to extract the feature of XSS payloads which captures word order information and map each payload to a feature vector. And then, we trained and tested the detection model using Long Short Term Memory (LSTM) recurrent neural networks. Experimental results show that the proposed XSS detection model based on deep learning achieves a precision rate of 99.5\% and a recall rate of 97.9\% in real dataset, which means that the novel approach can effectively identify XSS attacks.},
  file = {/Users/benjaminjoham/Zotero/storage/S52AVT5R/Fang et al. - 2018 - DeepXSS Cross Site Scripting Detection Based on D.pdf},
  isbn = {978-1-4503-6419-5},
}

@inproceedings{fazzini2015,
  title = {{{AutoCSP}}: {{Automatically Retrofitting CSP}} to {{Web Applications}}},
  shorttitle = {{{AutoCSP}}},
  booktitle = {2015 {{IEEE}}/{{ACM}} 37th {{IEEE International Conference}} on {{Software Engineering}}},
  author = {Fazzini, Mattia and Saxena, Prateek and Orso, Alessandro},
  year = {2015},
  month = may,
  volume = {1},
  pages = {336--346},
  issn = {1558-1225},
  doi = {10.1109/ICSE.2015.53},
  abstract = {Web applications often handle sensitive user data, which makes them attractive targets for attacks such as cross-site scripting (XSS). Content security policy (CSP) is a content-restriction mechanism, now supported by all major browsers, that offers thorough protection against XSS. Unfortunately, simply enabling CSP for a web application would affect the application's behavior and likely disrupt its functionality. To address this issue, we propose AutoCSP, an automated technique for retrofitting CSP to web applications. AutoCSP (1) leverages dynamic taint analysis to identify which content should be allowed to load on the dynamically-generated HTML pages of a web application and (2) automatically modifies the server-side code to generate such pages with the right permissions. Our evaluation, performed on a set of real-world web applications, shows that AutoCSP can retrofit CSP effectively and efficiently.},
  file = {/Users/benjaminjoham/Zotero/storage/USLFU6Q2/Fazzini et al. - 2015 - AutoCSP Automatically Retrofitting CSP to Web App.pdf;/Users/benjaminjoham/Zotero/storage/D52XMTQB/7194586.html},
  keywords = {Algorithm design and analysis,AutoCSP policy,Browsers,content security policy,Content security policy,cross-site scripting,CSP content-restriction mechanism,CSP retrofitting,dynamic taint analysis,dynamically-generated HTML pages,Heuristic algorithms,HTML,Internet,Security,security of data,server-side code modification,Servers,Web applications,Web pages,XSS protection}
}

@incollection{gupta2014,
  title = {{{BDS}}: {{Browser Dependent XSS Sanitizer}}},
  author = {Gupta, Shashank and Gupta, B B},
  year = {2014},
  month = nov,
  pages = {174--191},
  doi = {10.13140/2.1.4107.4245},
  file = {/Users/benjaminjoham/Downloads/gupta et al..2014 - BDS--Browser-Dependent-XSS-Sanitizer.pdf}
}

@article{gupta2015a,
  title = {Cross-{{Site Scripting}} ({{XSS}}) {{Abuse}} and {{Defense}}: {{Exploitation}} on {{Several Testing Bed Environments}} and {{Its Defense}}},
  shorttitle = {Cross-{{Site Scripting}} ({{XSS}}) {{Abuse}} and {{Defense}}},
  author = {Gupta, B. B. and Gupta, S. and Gangwar, S. and Kumar, M. and Meena, P. K.},
  year = {2015},
  month = apr,
  volume = {11},
  pages = {118--136},
  issn = {1553-6548, 2333-696X},
  doi = {10.1080/15536548.2015.1044865},
  file = {/Users/benjaminjoham/Zotero/storage/3TMQV2SS/Gupta et al. - 2015 - Cross-Site Scripting (XSS) Abuse and Defense Expl.pdf},
  journal = {Journal of Information Privacy and Security},
  number = {2}
}

@article{gupta2016a,
  title = {{{CSSXC}}: {{Context}}-Sensitive {{Sanitization Framework}} for {{Web Applications}} against {{XSS Vulnerabilities}} in {{Cloud Environments}}},
  author = {Gupta, Shashank and Gupta, Brij Bhooshan},
  year = {2016},
  month = jan,
  volume = {85},
  pages = {198--205},
  doi = {10.1016/J.PROCS.2016.05.211},
  file = {/Users/benjaminjoham/Zotero/storage/538JKEX5/Gupta, Gupta - 2016 - CSSXC Context-sensitive Sanitization Framework for Web Applications against XSS Vulnerabilities in Cloud Environme.pdf},
  journal = {Procedia Computer Science}
}

@article{gupta2017,
  title = {Cross-{{Site Scripting}} ({{XSS}}) Attacks and Defense Mechanisms: Classification and State-of-the-Art},
  author = {Gupta, Shashank and Gupta, Brij Bhooshan},
  year = {2017},
  month = jan,
  volume = {8},
  pages = {512--530},
  doi = {10.1007/s13198-015-0376-0},
  file = {/Users/benjaminjoham/Zotero/storage/LGYHAFIR/Cross-Site Scripting (XSS) attacks and defense mechanisms- classification and state-of-the-art.pdf},
  journal = {International Journal of System Assurance Engineering and Management},
  number = {1}
}

@article{gupta2018,
  title = {{{XSS}}-Secure as a Service for the Platforms of Online Social Network-Based Multimedia Web Applications in Cloud},
  author = {Gupta, Shashank and Gupta, B. B.},
  year = {2018},
  month = feb,
  volume = {77},
  pages = {4829--4861},
  issn = {1380-7501, 1573-7721},
  doi = {10.1007/s11042-016-3735-1},
  abstract = {This article presents a novel framework XSS-Secure, which detects and alleviates the propagation of Cross-Site Scripting (XSS) worms from the Online Social Network (OSN)based multimedia web applications on the cloud environment. It operates in two modes: training and detection mode. The former mode sanitizes the extracted untrusted variables of JavaScript code in a context-aware manner. This mode stores such sanitized code in sanitizer snapshot repository and OSN web server for further instrumentation in the detection mode. The detection mode compares the sanitized HTTP response (HRES) generated at the OSN web server with the sanitized response stored at the sanitizer snapshot repository. Any variation observed in this HRES message will indicate the injection of XSS worms from the remote OSN servers. XSS-Secure determines the context of such worms, perform the context-aware sanitization on them and finally sanitized HRES is transmitted to the OSN user. The prototype of our framework was developed in Java and integrated its components on the virtual machines of cloud environment. The detection and alleviation capability of our cloud-based framework was tested on the platforms of real world multimedia-based web applications including the OSN-based Web applications. Experimental outcomes reveal that our framework is capable enough to mitigate the dissemination of XSS worm from the platforms of non-OSN Web applications as well as OSN web sites with acceptable false negative and false positive rate.},
  file = {/Users/benjaminjoham/Zotero/storage/VWXINNDF/Gupta and Gupta - 2018 - XSS-secure as a service for the platforms of onlin.pdf},
  journal = {Multimed Tools Appl},
  number = {4}
}

@misc{helme2014,
  title = {Content {{Security Policy}} - {{An Introduction}}},
  author = {Helme, Scott},
  year = {2014},
  month = nov,
  abstract = {CSP allows you to whitelist sources of content the browser can load. An effective solution to XSS, it can be easily deployed and is widely supported.},
  file = {/Users/benjaminjoham/Zotero/storage/SF36LFXU/content-security-policy-an-introduction.html},
  howpublished = {https://scotthelme.co.uk/content-security-policy-an-introduction/},
  journal = {Scott Helme},
  note = {Library Catalog: scotthelme.co.uk}
}

@inproceedings{kirda2006,
  title = {Noxes: A Client-Side Solution for Mitigating Cross-Site Scripting Attacks},
  shorttitle = {Noxes},
  booktitle = {Proceedings of the 2006 {{ACM}} Symposium on {{Applied}} Computing  - {{SAC}} '06},
  author = {Kirda, Engin and Kruegel, Christopher and Vigna, Giovanni and Jovanovic, Nenad},
  year = {2006},
  pages = {330},
  publisher = {{ACM Press}},
  address = {{Dijon, France}},
  doi = {10.1145/1141277.1141357},
  abstract = {Web applications are becoming the dominant way to provide access to on-line services. At the same time, web application vulnerabilities are being discovered and disclosed at an alarming rate. Web applications often make use of JavaScript code that is embedded into web pages to support dynamic client-side behavior. This script code is executed in the context of the user's web browser. To protect the user's environment from malicious JavaScript code, a sandboxing mechanism is used that limits a program to access only resources associated with its origin site. Unfortunately, these security mechanisms fail if a user can be lured into downloading malicious JavaScript code from an intermediate, trusted site. In this case, the malicious script is granted full access to all resources (e.g., authentication tokens and cookies) that belong to the trusted site. Such attacks are called cross-site scripting (XSS) attacks.},
  file = {/Users/benjaminjoham/Zotero/storage/7V8FVJ9I/Kirda et al. - 2006 - Noxes a client-side solution for mitigating cross.pdf},
  isbn = {978-1-59593-108-5},
}

@article{kirda2009,
  title = {Client-Side Cross-Site Scripting Protection},
  author = {Kirda, Engin and Jovanovic, Nenad and Kruegel, Christopher and Vigna, Giovanni},
  year = {2009},
  month = oct,
  volume = {28},
  pages = {592--604},
  issn = {01674048},
  doi = {10.1016/j.cose.2009.04.008},
  abstract = {Web applications are becoming the dominant way to provide access to online services. At the same time, web application vulnerabilities are being discovered and disclosed at an alarming rate. Web applications often make use of JavaScript code that is embedded into web pages to support dynamic client-side behavior. This script code is executed in the context of the user's web browser. To protect the user's environment from malicious JavaScript code, browsers use a sand-boxing mechanism that limits a script to access only resources associated with its origin site. Unfortunately, these security mechanisms fail if a user can be lured into downloading malicious JavaScript code from an intermediate, trusted site. In this case, the malicious script is granted full access to all resources (e.g., authentication tokens and cookies) that belong to the trusted site. Such attacks are called cross-site scripting (XSS) attacks.},
  file = {/Users/benjaminjoham/Downloads/kirda2009.pdf},
  journal = {Computers \& Security},
  number = {7}
}

@article{liu2019,
  title = {A {{Survey}} of {{Exploitation}} and {{Detection Methods}} of {{XSS Vulnerabilities}}},
  author = {Liu, Miao and Zhang, Boyu and Chen, Wenbin and Zhang, Xunlai},
  year = {2019},
  volume = {7},
  pages = {182004--182016},
  issn = {2169-3536},
  doi = {10.1109/ACCESS.2019.2960449},
  abstract = {As web applications become more prevalent, web security becomes more and more important. Cross-site scripting vulnerability abbreviated as XSS is a kind of common injection web vulnerability. The exploitation of XSS vulnerabilities can hijack users' sessions, modify, read and delete business data of web applications, place malicious codes in web applications, and control victims to attack other targeted servers. This paper discusses classification of XSS, and designs a demo website to demonstrate attack processes of common XSS exploitation scenarios. The paper also compares and analyzes recent research results on XSS detection, divides them into three categories according to different mechanisms. The three categories are static analysis methods, dynamic analysis methods and hybrid analysis methods. The paper classifies 30 detection methods into above three categories, makes overall comparative analysis among them, lists their strengths and weaknesses and detected XSS vulnerability types. In the end, the paper explores some ways to prevent XSS vulnerabilities from being exploited.},
  file = {/Users/benjaminjoham/Zotero/storage/7PA3KF9J/Liu et al. - 2019 - A Survey of Exploitation and Detection Methods of .pdf;/Users/benjaminjoham/Zotero/storage/79CDM2WE/8935148.html},
  journal = {IEEE Access},
  keywords = {cross-site scripting vulnerability,dynamic analysis methods,hybrid analysis methods,injection web vulnerability,Internet,program diagnostics,security of data,static analysis methods,Vulnerability detection,vulnerability exploitation,web applications,web security,XSS,XSS detection,XSS exploitation scenarios,XSS vulnerability types},
  note = {Conference Name: IEEE Access}
}

@inproceedings{mahmoud2017,
  title = {A Comparative Analysis of {{Cross Site Scripting}} ({{XSS}}) Detecting and Defensive Techniques},
  booktitle = {2017 {{Eighth International Conference}} on {{Intelligent Computing}} and {{Information Systems}} ({{ICICIS}})},
  author = {Mahmoud, S. K. and Alfonse, M. and Roushdy, M. I. and Salem, A. M.},
  year = {2017},
  month = dec,
  pages = {36--42},
  doi = {10.1109/INTELCIS.2017.8260024},
  abstract = {Now the web applications are highly useful and powerful for usage in most fields such as finance, e-commerce, healthcare and more, so it must be well secured. The web applications may contain vulnerabilities, which are exploited by attackers to steal the user's credential. The Cross Site Scripting (XSS) attack is a critical vulnerability that affects on the web applications security. XSS attack is an injection of malicious script code into the web application by the attacker in the client-side within user's browser or in the server-side within the database, this malicious script is written in JavaScript code and injected within untrusted input data on the web application. This study discusses the XSS attack, its taxonomy, and its incidence. In addition, the paper presents the XSS mechanisms used to detect and prevent the XSS attacks.},
  file = {/Users/benjaminjoham/Zotero/storage/YXLW8TJL/Mahmoud et al. - 2017 - A comparative analysis of Cross Site Scripting (XS.pdf;/Users/benjaminjoham/Zotero/storage/PUVT68YT/8260024.html},
  keywords = {Browsers,Cross Site Scripting (XSS),cross site scripting attack,defensive techniques,DOM-base attack,injection code,Internet,JavaScript code,Malicious JavaScript,malicious script code,online front-ends,Reflected attack,security of data,Servers,Stored attack,Tools,web application security,Web application security,Web pages,web security,XSS attack,XSS mechanisms,XSS vulnerability}
}

@inproceedings{mohammadi2017,
  title = {Detecting {{Cross}}-{{Site Scripting Vulnerabilities}} through {{Automated Unit Testing}}},
  booktitle = {2017 {{IEEE International Conference}} on {{Software Quality}}, {{Reliability}} and {{Security}} ({{QRS}})},
  author = {Mohammadi, Mahmoud and Chu, Bill and Lipford, Heather Richter},
  year = {2017},
  month = jul,
  pages = {364--373},
  doi = {10.1109/QRS.2017.46},
  abstract = {The best practice to prevent Cross Site Scripting (XSS) attacks is to apply encoders to sanitize untrusted data. To balance security and functionality, encoders should be applied to match the web page context, such as HTML body, JavaScript, and style sheets. A common programming error is the use of a wrong encoder to sanitize untrusted data, leaving the application vulnerable. We present a security unit testing approach to detect XSS vulnerabilities caused by improper encoding of untrusted data. Unit tests for the XSS vulnerability are automatically constructed out of each web page and then evaluated by a unit test execution framework. A grammar-based attack generator is used to automatically generate test inputs. We evaluate our approach on a large open source medical records application, demonstrating that we can detect many 0-day XSS vulnerabilities with very low false positives, and that the grammar-based attack generator has better test coverage than industry best practices.},
  file = {/Users/benjaminjoham/Zotero/storage/4WH6HZLZ/Mohammadi et al. - 2017 - Detecting Cross-Site Scripting Vulnerabilities thr.pdf;/Users/benjaminjoham/Zotero/storage/IWVIXPTR/8009940.html},
  keywords = {attack generation,automated unit testing,automatic programming,Browsers,cross-site scripting vulnerabilities detection,encoder,Encoding,grammar-based attack generator,grammars,HTML,injection attacks,Java,open source medical records application,program analysis,program diagnostics,program testing,programming error,public domain software,security,Security,security of data,security unit testing approach,Testing,unit test execution framework,unit testing,untrusted data sanitization,Web page context,Web pages,XSS attacks,XSS vulnerability}
}

@inproceedings{pan2016,
  title = {{{CSPAutoGen}}: {{Black}}-Box {{Enforcement}} of {{Content Security Policy}} upon {{Real}}-World {{Websites}}},
  shorttitle = {{{CSPAutoGen}}},
  booktitle = {Proceedings of the 2016 {{ACM SIGSAC Conference}} on {{Computer}} and {{Communications Security}} - {{CCS}}'16},
  author = {Pan, Xiang and Cao, Yinzhi and Liu, Shuangping and Zhou, Yu and Chen, Yan and Zhou, Tingzhe},
  year = {2016},
  pages = {653--665},
  publisher = {{ACM Press}},
  address = {{Vienna, Austria}},
  doi = {10.1145/2976749.2978384},
  abstract = {Content security policy (CSP)\textemdash{}which has been standardized by W3C and adopted by all major commercial browsers\textemdash{}is one of the most promising approaches for defending against cross-site scripting (XSS) attacks. Although client-side adoption of CSP is successful, server-side adoption is far behind the client side: according to a large-scale survey, less than 0.002\% of Alexa Top 1M websites enabled CSP.},
  file = {/Users/benjaminjoham/Zotero/storage/V3ISXEQS/Pan et al. - 2016 - CSPAutoGen Black-box Enforcement of Content Secur.pdf},
  isbn = {978-1-4503-4139-4},
}

@inproceedings{parameshwaran2015,
  title = {{{DexterJS}}: Robust Testing Platform for {{DOM}}-Based {{XSS}} Vulnerabilities},
  shorttitle = {{{DexterJS}}},
  booktitle = {Proceedings of the 2015 10th {{Joint Meeting}} on {{Foundations}} of {{Software Engineering}} - {{ESEC}}/{{FSE}} 2015},
  author = {Parameshwaran, Inian and Budianto, Enrico and Shinde, Shweta and Dang, Hung and Sadhu, Atul and Saxena, Prateek},
  year = {2015},
  pages = {946--949},
  publisher = {{ACM Press}},
  address = {{Bergamo, Italy}},
  doi = {10.1145/2786805.2803191},
  abstract = {DOM-based cross-site scripting (XSS) is a client-side vulnerability that pervades JavaScript applications on the web, and has few known practical defenses. In this paper, we introduce DEXTERJS, a testing platform for detecting and validating DOM-based XSS vulnerabilities on web applications. DEXTERJS leverages source-tosource rewriting to carry out character-precise taint tracking when executing in the browser context \textemdash{} thus being able to identify vulnerable information flows in a web page. By scanning a web page, DEXTERJS produces working exploits that validate DOM-based XSS vulnerability on the page. DEXTERJS is robust, has been tested on Alexa's top 1000 sites, and has found a total of 820 distinct zero-day DOM-XSS confirmed exploits automatically.},
  file = {/Users/benjaminjoham/Zotero/storage/VCCUK5SV/Parameshwaran et al. - 2015 - DexterJS robust testing platform for DOM-based XS.pdf},
  isbn = {978-1-4503-3675-8},
}

@inproceedings{steinhauser2016,
  title = {{{JSPChecker}}: {{Static Detection}} of {{Context}}-{{Sensitive Cross}}-{{Site Scripting Flaws}} in {{Legacy Web Applications}}},
  shorttitle = {{{JSPChecker}}},
  booktitle = {Proceedings of the 2016 {{ACM Workshop}} on {{Programming Languages}} and {{Analysis}} for {{Security}} - {{PLAS}}'16},
  author = {Steinhauser, Antonin and Gauthier, Fran{\c c}ois},
  year = {2016},
  pages = {57--68},
  publisher = {{ACM Press}},
  address = {{Vienna, Austria}},
  doi = {10.1145/2993600.2993606},
  abstract = {JSPChecker is a static analysis tool that detects contextsensitive cross-site scripting vulnerabilities in legacy web applications. While cross-site scripting flaws can be mitigated through sanitisation, a process that removes dangerous characters from input values, proper sanitisation requires knowledge about the output context of input values. Indeed, web pages are built using a mix of different languages (e.g. HTML, CSS, JavaScript and others) that call for different sanitisation routines. Context-sensitive crosssite scripting vulnerabilities occur when there is a mismatch between sanitisation routines and output contexts.},
  file = {/Users/benjaminjoham/Zotero/storage/CSH46UT9/Steinhauser and Gauthier - 2016 - JSPChecker Static Detection of Context-Sensitive .pdf},
  isbn = {978-1-4503-4574-3},
}

@inproceedings{su2006,
  title = {The {{Essence}} of {{Command Injection Attacks}} in {{Web Applications}}},
  booktitle = {{{ACM SIGPLAN}}-{{SIGACT}} Symposium on {{Principles}} of Programming Languages},
  author = {Su, Zhendong and Wassermann, Gary},
  year = {2006},
  month = jan,
  volume = {33},
  pages = {372--382},
  publisher = {{ACM}},
  doi = {10.1145/1111037.1111070},
  abstract = {Web applications typically interact with a back-end database to retrieve persistent data and then present the data to the user as dynamically generated output, such as HTML web pages. However, this interaction is commonly done through a low-level API by dynamically constructing query strings within a general-purpose programming language, such as Java. This low-level interaction is ad hoc because it does not take into account the structure of the output language. Accordingly, user inputs are treated as isolated lexical entities which, if not properly sanitized, can cause the web application to generate unintended output. This is called a command injection attack, which poses a serious threat to web application security. This paper presents the first formal definition of command injection attacks in the context of web applications, and gives a sound and complete algorithm for preventing them based on context-free grammars and compiler parsing techniques. Our key observation is that, for an attack to succeed, the input that gets propagated into the database query or the output document must change the intended syntactic structure of the query or document. Our definition and algorithm are general and apply to many forms of command injection attacks. We validate our approach with SQLCHECK, an implementation for the setting of SQL command injection attacks. We evaluated SQLCHECK on real-world web applications with systematically compiled real-world attack data as input. SQLCHECK produced no false positives or false negatives, incurred low runtime overhead, and applied straightforwardly to web applications written in different languages.},
  file = {/Users/benjaminjoham/Zotero/storage/3F9LG8FB/Su and Wassermann - The Essence of Command Injection Attacks in Web Ap.pdf},
  series = {{{POPL}} '06}
}

@inproceedings{wang2016,
  title = {A {{New Cross}}-{{Site Scripting Detection Mechanism Integrated}} with {{HTML5}} and {{CORS Properties}} by {{Using Browser Extensions}}},
  booktitle = {2016 {{International Computer Symposium}} ({{ICS}})},
  author = {Wang, Chih-Hung and Zhou, Yi-Shauin},
  year = {2016},
  month = dec,
  pages = {264--269},
  doi = {10.1109/ICS.2016.0060},
  abstract = {Cross site scripting (XSS) is a kind of common attack nowadays. The attack patterns with the new technical like HTML5 that makes detection task getting harder and harder. In this paper, we focus on the browser detection mechanism integrated with HTML5 and CORS properties to detect XSS attacks with the rule based filter by using browser extensions. Further, we also present a model of composition pattern estimation system which can be used to judge whether the intercepted request has malicious attempts or not. The experimental results show that our approach can reach high detection rate by tuning our system through some frequently used attack sentences and testing it with the popular tool-kits: XSSer developed by OWASP.},
  file = {/Users/benjaminjoham/Zotero/storage/C45WCDXD/Wang and Zhou - 2016 - A New Cross-Site Scripting Detection Mechanism Int.pdf;/Users/benjaminjoham/Zotero/storage/ULV6WZWH/7858482.html},
  keywords = {browser extension,browser extensions,Browsers,CORS,cross origin resource shearing (CORS),cross-site scripting (XSS),cross-site scripting detection mechanism,Databases,HTML5,hypermedia markup languages,Malware,OWASP,Security,security of data,Servers,Uniform resource locators,web security,XSS attack detection}
}

@article{wang2018,
  title = {{{TT}}-{{XSS}}: {{A}} Novel Taint Tracking Based Dynamic Detection Framework for {{DOM Cross}}-{{Site Scripting}}},
  shorttitle = {{{TT}}-{{XSS}}},
  author = {Wang, Ran and Xu, Guangquan and Zeng, Xianjiao and Li, Xiaohong and Feng, Zhiyong},
  year = {2018},
  month = aug,
  volume = {118},
  pages = {100--106},
  issn = {07437315},
  doi = {10.1016/j.jpdc.2017.07.006},
  abstract = {Most work on DOM Cross-Site Scripting (DOM-XSS) detection methods can be divided into three kinds: black-box fuzzing, static analysis, and dynamic analysis. However, black-box fuzzing and static analysis suffer much from high false negative rates and high false positive rates respectively. Current dynamic analysis is complex and expensive, though it can obtain more efficient results. In this paper, we propose a dynamic detection framework (TT-XSS) for DOM-XSS by means of taint tracking at client side. We rewrite all JavaScript features and DOM APIs to taint the rendering process of browsers. To this end, new data types and methods are presented to extend the semantic description ability of the original data structure, based on which we can analyze the taint traces through tainting all sources, sinks and transfer processes during pages parsing. In this way, attack vectors are derived to verify the vulnerabilities automatically. Compared to AWVS 10.0, our framework detects more 1.8\% vulnerabilities, and it can generate the corresponding attack vectors to verify 9.1\% vulnerabilities automatically.},
  file = {/Users/benjaminjoham/Zotero/storage/FQ4ADTQ3/Wang et al. - 2018 - TT-XSS A novel taint tracking based dynamic detec.pdf},
  journal = {Journal of Parallel and Distributed Computing},
}



@article{stock2014,
  title = {Precise Client-Side Protection against {{DOM}}-Based {{Cross}}-{{Site Scripting}}},
  author = {Stock, Ben and Lekies, Sebastian and Mueller, Tobias},
  year = {2014},
  pages = {17},
  abstract = {The current generation of client-side Cross-Site Scripting filters rely on string comparison to detect request values that are reflected in the corresponding response's HTML. This coarse approximation of occurring data flows is incapable of reliably stopping attacks which leverage nontrivial injection contexts. To demonstrate this, we conduct a thorough analysis of the current state-of-the-art in browser-based XSS filtering and uncover a set of conceptual shortcomings, that allow efficient creation of filter evasions, especially in the case of DOM-based XSS. To validate our findings, we report on practical experiments using a set of 1,602 real-world vulnerabilities, achieving a rate of 73\% successful filter bypasses. Motivated by our findings, we propose an alternative filter design for DOM-based XSS, that utilizes runtime taint tracking and taint-aware parsers to stop the parsing of attackercontrolled syntactic content. To examine the efficiency and feasibility of our approach, we present a practical implementation based on the open source browser Chromium. Our proposed approach has a low false positive rate and robustly protects against DOM-based XSS exploits.},
  file = {/Users/benjaminjoham/Zotero/storage/R8M5TD46/Stock et al. - Precise client-side protection against DOM-based C.pdf},
}




@techreport{kfv2019,
  title = {{Cybersicherheit als Chance}},
  author = {KFV},
  year = {2019},
  month = dec,
}



@phdthesis{wolf2012,
  title = {{WebForensik - Forensische Analyse von Apache HTTPD Logfiles}},
  author = {Wolf, Christian and M{\"u}ller, Jens and Wolf, Michael},
  year = {2012},
  month = apr,
  address = {{Bochum}},
  abstract = {Im Studienprojekt WebForensik sollen Methoden zur Erkennung von Angriffen gegen
Webserver entwickelt werden. Als Datengrundlage dienen die Logfiles des Apache
HTTP Servers. Daraus soll eine automatisierte, forensische Analyse erfolgen, die Zeitpunkt und Art von Angriffen sowie deren Auswirkung auf das betroffene System
m{\"o}glichst zuverl{\"a}ssig erkennt. Ferner soll aufgezeigt werden, wie diese Auswertung
durch die Verwendung von Anonymisierungsdiensten und Methoden zur Verschleierung erschwert wird, und welche Gegenma{\ss}nahmen hierzu wiederum existieren.},
  file = {/Users/benjaminjoham/Zotero/storage/78GQ9VJX/Wolf et al. - WebForensik - Forensische Analyse von Apache HTTPD.pdf},
  school = {Horst-G{\"o}rtz Institut Ruhr-Universit{\"a}t Bochum}
}



@misc{schuring2017,
  title = {{Cross-Site-Scripting \textendash{} Wie Hacker deine Seite kapern}},
  author = {Sch{\"u}ring, Tobias},
  year = {2017},
  month = jul,
  abstract = {Mit Cross-Site-Scripting, aka XSS, kapern Hacker deine Seite und missbrauchen sie f{\"u}r ihre Zwecke. Wir zeigen wie gef{\"a}hrlich die Angriffe sind und wie sie funktionieren.},
  chapter = {Sicherheit},
  file = {/Users/benjaminjoham/Zotero/storage/MBN6CSJ3/cross-site-scripting.html},
  journal = {RAIDBOXES},
  note = {Library Catalog: raidboxes.io}
}

@misc{owasp2020,
  title = {{{DOM}} Based {{XSS Prevention}}},
  author = {{OWASP}},
  year = {2020},
  file = {/Users/benjaminjoham/Zotero/storage/NCK48U8G/DOM_based_XSS_Prevention_Cheat_Sheet.html},
  howpublished = {https://cheatsheetseries.owasp.org/cheatsheets/DOM\_based\_XSS\_Prevention\_Cheat\_Sheet.html}
}


@article{hydara2015a,
  title = {Current State of Research on Cross-Site Scripting ({{XSS}}) \textendash{} {{A}} Systematic Literature Review},
  author = {Hydara, Isatou and Sultan, Abu Bakar Md. and Zulzalil, Hazura and Admodisastro, Novia},
  year = {2015},
  month = feb,
  volume = {58},
  pages = {170--186},
  doi = {10.1016/J.INFSOF.2014.07.010},
  file = {/Users/benjaminjoham/Zotero/storage/SK89YC2P/Current state of research on cross-site scripting.pdf},
  journal = {Information and Software Technology}
}

@article{abbosh2019,
  title = {Securing the {{Digital Economy}}: {{Reinventing}} the {{Internet}} for {{Trust}}},
  author = {Abbosh, Omar and Bissell, Kelly},
  year = {2019},
  pages = {49},
  file = {/Users/benjaminjoham/Zotero/storage/7KZWL5CQ/Abbosh and Bissell - Securing the Digital Economy Reinventing the Inte.pdf},
}

@misc{eilers2015,
  title = {Cross-{{Site Scripting}} Verhindern, Ganz Allgemein},
  author = {Eilers, Dipl.-Inform. Carsten},
  year = {2015},
  month = aug,
  file = {/Users/benjaminjoham/Zotero/storage/HX3FELXN/696-Cross-Site-Scripting-verhindern,-ganz-allgemein.html},
  howpublished = {https://www.ceilers-news.de/serendipity/696-Cross-Site-Scripting-verhindern,-ganz-allgemein.html}
}



@misc{swaswatigoswami2017,
  title = {An {{Unsupervised Method}} for {{Detection}} of {{XSS Attack}}},
  author = {{Swaswati Goswami} and {Nazrul Hoque} and {Dhruba K. Bhattacharyya} and {Jugal Kalita}},
  year = {2017},
  month = sep,
  abstract = {Cross-site scripting (XSS) is a code injection attack that allows an attacker to execute malicious script in another user's browser. Once the attacker gains control over the Website vulnerable to XSS attack, it can perform actions like cookie-stealing, malware-spreading, session-hijacking and malicious redirection. Malicious JavaScripts are the most conventional ways of performing XSS attacks. Al- though several approaches have been proposed, XSS is still a live problem since it is very easy to implement, but difficult to detect. In this paper, we propose an effective approach for XSS attack detection. Our method focuses on balancing the load between client and the server. Our method performs an initial checking in the client side for vulnerability using divergence measure. If the suspicion level exceeds beyond a threshold value, then the request is discarded. Otherwise, it is forwarded to the proxy for further processing. In our approach we intro- duce an attribute clustering method supported by rank aggregation technique to detect confounded JavaScripts. The approach is validated using real life data.},
  file = {/Users/benjaminjoham/Zotero/storage/JEMX2XNV/Swaswati Goswami et al. - 2017 - An Unsupervised Method for Detection of XSS Attack.pdf},
}



@misc{klein2005,
  title = {{{DOM Based Cross Site Scripting}} or {{XSS}} of the {{Third Kind}}},
  author = {Klein, Amit},
  year = {2005},
  month = jul,
  file = {/Users/benjaminjoham/Zotero/storage/26XM8TZU/071105.html},
  howpublished = {http://www.webappsec.org/projects/articles/071105.html}
}



@misc{eilers2015a,
  title = {Cross-{{Site Scripting}} Im {{\"Uberblick}}, {{Teil}} 4: {{DOM}}-Basiertes {{XSS}} | {{Dipl}}.-{{Inform}}. {{Carsten Eilers}}},
  author = {Eilers, Dipl.-Inform. Carsten},
  year = {2015},
  month = mar,
  file = {/Users/benjaminjoham/Zotero/storage/U28H96QE/633-Cross-Site-Scripting-im-UEberblick,-Teil-4-DOM-basiertes-XSS.html},
  howpublished = {https://www.ceilers-news.de/serendipity/633-Cross-Site-Scripting-im-UEberblick,-Teil-4-DOM-basiertes-XSS.html}
}


