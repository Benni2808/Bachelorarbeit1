% Mit über eine Milliarde Websites weltweit und über 4 Milliarden Internetnutzern weltweit bietet diese Anzahl potentiellen Kriminellen eine große Auswahl an Opfern. Cross-Site-Scripting (XSS) ist eines der weitverbreitesten Sicherheitsriken im Web, wie wir es kennen. Ziel dieser Arbeit war es herauszufinden, ob es eine bewährte Methode gibt, XSS Schwachstellen von Web-Anwendungen zu erkennen und ob man diese sogar ganz verhindern kann.

% Zunächst wurden die drei Arten von XSS, stored, reflected und DOM-based untersucht. Im weiteren Schritt wurden Möglichkeiten zum Erkennen diverser XSS Schwachstellen verglichen. Abschließend wurden Methoden zum allgemeinen Verhindern von solchen Schwachstellen ermittelt. Die Hauptergebnisse waren, dass es bereits eine Vielzahl von Frameworks und Methoden gibt um bereits bestehende XSS Schwachstellen ausfindig zu machen, aber nur wenige Versuche diese bereits im vorhinein zu eliminieren. Jedoch gibt es einen guten Ansatz die Web-Anwendung vor XSS Anfälligkeiten zu schützen mit der Hilfe Content-Security-Policy (CSP) Headern.