Transporting data, especially sensitive data through the Internet is a part of today’s Web Ap-plications as Kirda et al. (2009) mentioned in their paper.  Typically Web Applications interactwith back-end databases to receive data which will later be parsed to the user as dynamicallygenerated output as Su and Wassermann (2006) stated.Hydara, Sultan, et al. (2015) mentioned that Cross Site Scripting (XSS) vulnerabilities havefirst been discovered approximately around 1990 in the early days of the World Wide Web.According to Kirda et al. (2009), Cross Site Scripting can roughly be divided into three mainclasses.   The  first  class,  stored  XSS,  is  based  on  attacks  from  the  Web  server.   This  meansmalicious code by the attacker is injected and stored on a Web server and will be executed whensomebody accesses the Website.The second class, reflected XSS, basically is the equivalent of stored XSS with the smalldifference that the code is reflected from the Web server. This can happen for for instance whena user is manipulated into clicking on a link in an E-Mail.The third class and the one this thesis will deal the most with, DOM-based XSS. Sarmah,Bhattacharyya, and Kalita (2018) said, that DOM-based XXS attacks differ significantly fromthe above because it is possible to execute a malicious script with help of the interpreter in thebrowser. Furthermore, it can not be noted in the response. According to Sarmah, Bhattacharyya,and Kalita (2018), it can either be found by structuring the Document-Object-Model (DOM) ore.g. at runtime, when a Web page is loaded.In order to detect such XSS attacks,  Wassermann and Su (2008) had one of the first ap-proaches on the detection of server-side appearing Cross Site Scripting (XSS) vulnerabilities.Client-side-wise the first solution has been presented from Kirda et al. (2009). They developeda solution, based on personal Web Firewall Applications, to mitigate cross-site scripting attacks.This tool is called Noxes.  There are even more detection approaches, but due to the length ofthis thesis some will be skipped.This Thesis will be going into detail about the different types of Cross Site Scripting, howsuch attacks can be recognized in the first stadium and most important how to prevent yourself1
and your Web Applications for Cross Site Scripting (XSS) specialized for Single Page Appli-cations.In the first part of the paper, I am going to explain the various types of Cross Site Scripting(XSS). From there we will dig into finer details about client-side Cross Site Scripting with aspecial focus on DOM BASED XSS as Shrivastava,  Choudhary,  and Kumar (2016) pointedout.The second part will follow up with detection approaches that have been done e.g. Sarmah,Bhattacharyya, and Kalita (2018).Continuing with Nithya, Pandian, and Malarvizhi (2015) to give you a deeper knowledgeabout prevention approaches.The goal of this thesis is to lead you on a proper way to handle security issues referring toCross Site Scripting in your following Single Page Application development.