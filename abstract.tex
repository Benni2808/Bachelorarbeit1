With over one billion websites worldwide and over 4 billion Internet users worldwide, this number offers potential criminals a wide range of victims. Cross-site scripting (XSS) is one of the most widespread security issue on the web as we know it. The goal of this thesis was to find out if there is a proven method to detect XSS vulnerabilities of web applications and if it is even possible to prevent them completely.

First, the three types of XSS, stored, reflected and DOM-based were investigated. In the next step, possibilities for detecting various XSS vulnerabilities were compared. Finally, methods for the general prevention of such vulnerabilities were identified. The main results were that there are already a lot of frameworks and methods to detect XSS vulnerabilities, but only few attempts to eliminate them in advance. However, there is a good approach to protect the web application against XSS vulnerabilities with the help of Content Security Policy (CSP) headers.