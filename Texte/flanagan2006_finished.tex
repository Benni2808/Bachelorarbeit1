Nach \textcite[1]{kirda2009} kann vom Interpreter automatisch ausgeführter JavaScript Code einen möglichen Raum für Angriffe gegen die Benutzerumgebung darstellen. Eine sichere Ausführung von JavaScript Code basiert auf dem Prinzip einer Sandbox\footnote[2]{https://de.wikipedia.org/wiki/Sandbox}. Diese erlaubt es dem Code nur gewisse Operationen auszuführen und gewährt nur begrenzten Zugriff auf Ressourcen im Web-Browser. Ebenso werden JavaScript Programme von verschiedenen Seiten mit einem Abschottungsmechanismus, der "same origin policy"\footnote[2]{https://de.wikipedia.org/wiki/Same-Origin-Policy} geschützt. Diese erlaubt es dem Programm nur auf Ressourcen innerhalb seines Ursprungs zuzugreifen.