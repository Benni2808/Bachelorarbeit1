\section*{Concept}

Transporting data, especially sensitive data through the Internet is a part of today's Web Applications as \textcite[]{Kirda2009} mentioned in their paper. Typically Web Applications interact with back-end databases to receive data which will later be parsed to the user as dynamically generated output as \textcite[]{Su2006} stated.

\textcite[]{Hydara2015} mentioned that Cross Site Scripting (XSS) vulnerabilities have first been discovered approximately around 1990 in the early days of the World Wide Web.
According to \textcite[]{Kirda2009}, Cross Site Scripting can roughly be divided into three main classes. The first class, stored XSS, is based on attacks from the Web server. This means malicious code by the attacker is injected and stored on a Web server and will be executed when somebody accesses the Website.

The second class, reflected XSS, basically is the equivalent of stored XSS with the small difference that the code is reflected from the Web server. This can happen for for instance when a user is manipulated into clicking on a link in an E-Mail.

The third class and the one this thesis will deal the most with, DOM-based XSS. \textcite[]{Sarmah2018} said, that DOM-based XXS attacks differ significantly from the above because it is possible to execute a malicious script with help of the interpreter in the browser.
Furthermore, it can not be noted in the response. According to \textcite[]{Sarmah2018}, it can either be found by structuring the Document-Object-Model (DOM) or e.g. at runtime, when a Web page is loaded. 

In order to detect such XSS attacks,
\textcite[]{Wassermann2008} had one of the first approaches on the detection of server-side appearing Cross Site Scripting (XSS) vulnerabilities. Client-side-wise the first solution has been presented from \textcite[]{Kirda2009}. They developed a solution, based on personal Web Firewall Applications, to mitigate cross-site scripting attacks. This tool is called Noxes. There are even more detection approaches, but due to the length of this thesis some will be skipped.



%According to \textcite[]{Hydara2015} Cross Site Scripting (XSS) vulnerabilities have first been discovered in the early 1990 in the early days of the World Wide Web.



This Thesis will be going into detail about the different types of Cross Site Scripting, how such attacks can be recognized in the first stadium and most important how to prevent yourself and your Web Applications for Cross Site Scripting (XSS) specialized for Single Page Applications.

In the first part of the paper, I am going to explain the various types of Cross Site Scripting (XSS). From there we will dig into finer details about client-side Cross Site Scripting with a special focus on DOM BASED XSS as \textcite[]{Shrivastava2016} pointed out.

The second part will follow up with detection approaches that have been done e.g. \textcite[]{Sarmah2018}.

Continuing with \textcite[]{Nithya2015} to give you a deeper knowledge about prevention approaches.

The goal of this thesis is to lead you on a proper way to handle security issues referring to Cross Site Scripting in your following Single Page Application development.